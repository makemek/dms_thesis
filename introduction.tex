\chapter{Introduction}

\section{Motivation}
\IC (IC) is a small organization located in \kmitl (KMITL). 
IC have been up and running well for 6 years. 
Around 25 employees are working here. 
Their main objective is to provide educational services to students. 
Right now, they have a trouble with paper clutter. 
Even though they have organized documents into category. 
The company takes countless hours or even weeks to get a desired document. 
This problem hinders employee's productivity when they need to retrieve many documents. 
Another problem is losing track of document during its workflow execution. 
A workflow is a series of repeatable steps performed in a sequential manner to reach a goal.
Each type of workflow usually associates with a set of actions. 
A person who receive a document will execute some actions depending on their responsibility.
They have to execute them sequentially. 
Sometimes, employee forgets on current state of document's workflow. 
They have to recheck them again.

IC would like to prevent these problems before it goes out of control. 
We would like to help them. 
Our proposed solution is to move physical papers to digital format. 
Meaning that they have to scan documents not owned by IC through a scanner. 
We will create a system that provide ability to search and to track documents.

\section{Objective}
% Aims
We are working on \dms.
Because we want to provide a system to manage documents within IC.
Once we do, IC can manage and track documents more efficient, thus, increase organization's productivity.

% list of our primary objectives
Our primary objectives are
\begin{enumerate}
\item To track document's current state during its workflow execution.
\item To report other related or dependent documents required by its originator.
\item To store and retrieve digital documents that complete its workflow from one centralized IC's server.
\end{enumerate}

\section{Scope}
We are going to develop a web-based application hosted on an IC's server.
Only people within IC and granted external users can access the system.
To be specific, there are 4 types of users.
\begin{enumerate}
\item IC's staffs.
%TODO list all employee's position who invovled in this system
\item KMITL lecturers who is assigned to IC.
\item Stdents who is currently enroll in IC, KMITL.
\item External users who is allowed to access the system by IC.
\end{enumerate}

The system have to manage only document distributed officially by IC.
This also includes other attachments from external sources given that IC requires them.
No personal document involved.