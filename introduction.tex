\chapter{Introduction}

\section{Motivation}
\gls{ic} is a faculty of \gls{kmitl}. 
IC has been up and running well for 6 years.
About 25 employees are working here. %TODO add number of students, lecturers
Their main objective is to provide educational services to students. 
Right now, they are having 2 major problems involved with their accumulating documents.
First, the organization takes countless hours to store and retrieve documents even though they have organized documents into categories.
By simply retrieving a document, as trivial as it seems, can take hours or even weeks. 
This problem hinders employee's productivity when they need to retrieve many documents. 
A second problem is losing track of document during its workflow execution. 
A workflow is a series of repeatable steps performed in a sequential manner to reach a goal.
It provides specification, execution and control of business processes \cite{Jablonski:1996:WMM}. 
All of documents can be as aligned to workflows in order to control business process in document management system \cite{DBLP:journals/corr/AsiliT14}.
A person who hold responsibility of a document will execute actions based on their assigned roles.
The problem is that some IC's workflows are so complicated that they are difficult to keep track of.
There are many actions and conditions to execute.
Many People from both inside and outside of the organization may involve during workflow's execution.
Supplementary documents, or so called attachments, may need to be attached with original document in order to pass some certain guidelines.
As a result, IC can't certainly identify who is currently involved and what actions they must do with that document.
They also can't track attachments effectively causing delay in archiving.

IC would like to prevent these problems before they go out of control. 
Our proposed solution is to manage, store, and retrieve documents digitally using a computer software.
The employee will use less time to search and be less error-prone.
We will create a system that provide ability to search and to track documents.
We are working on \dms.
Because we want to provide a system to manage documents within IC.
Once we do, IC can manage and track documents more efficient, thus, increase organization's productivity.

\section{Objective}
% list of our primary objectives
Our primary objectives are
\begin{enumerate}
\item To acquire IC's workflow model associated with their official documents.\footnote{
	Although workflow is not our main focus, we work closely with other researchers on dynamic workflow system. %TODO cite Boy's thesis at proj's name
	Not only we share a similar objective, but also we have to link our system with them.
	We need their general workflow models to achieve this objective.
	}
\item To track and to report document's current action and attachment during workflow's execution.
\item To store and to retrieve digital documents that complete its workflow.
\end{enumerate}

\section{Scope}
We are going to develop a web-based application hosted on an IC's server.
Only people within IC and granted external users can access the system.
To be specific, there are 4 types of users.
\begin{enumerate}
\item IC's staffs.
%TODO list all employee's position who invovled in this system
\item KMITL lecturers who is assigned to IC.
\item Students who is currently enrolling in IC, KMITL.
\item External users who is allowed to access the system by IC.
\end{enumerate}

Our primary focus is the end-product of a document workflow.
They are filled document from a predefined template with other attachments included optionally.
The system have to manage only document distributed officially by IC.
This also includes other attachments from external sources given that IC requires them.
No personal document involved because they are not a property of IC.
This thesis conducts only on the following 7 types of documents.
\begin{enumerate}
\item \label{quality-manual} Quality Manual \hfill \\
A document on policies management.
It describes the following information about the organization.
\begin{multicols}{2}
\begin{itemize}
\item Management strategy
\item Scope
\item Goals
\item Work procedure
\item Management structure
\item Objectives
\item Responsibilities
\end{itemize}
\end{multicols}

\item \label{doc-type:work-prod} Work Procedure \hfill \\
A document describes details on working procedures, departments or persons who hold responsibilities for documents.

\item \label{doc-type:work-ins} Work Instruction \hfill \\
A document indicates employee's roles and responsibilities.
The employee can work to meet organization's needs.

\item Forms \label{doc-type:form} \hfill \\
Forms use to collect information or work results.
They reference either type \ref{doc-type:work-prod} or \ref{doc-type:work-ins}.

\item Records \label{doc-type:record} \hfill \\
A record is an already-filled form.

\item Supported Document \hfill \\
Other IC's document that are not type \ref{quality-manual}, \ref{doc-type:work-prod}, \ref{doc-type:work-ins}, \ref{doc-type:form}, and \ref{doc-type:record}

\item External Document \hfill \\
A document receives from other departments in order to proceed some tasks inside IC.
\end{enumerate}

\section{Project plan}
The project starts at August 18th, 2015.
We expect to deliver it on March 25th, 2016.
For our project planing, we divide our work into 4 phases.
%TODO descibe them
\begin{enumerate}
\item Plan and research (18/08/15 -- 30/10/15) \hfill \\
The first phase is to gather user's requirements by interviewing.
We are going to interview IC's vice dean because he is a client who came up with this project.
Then, we will move on to interviewing IC's employees asking about document related problem they encountered.
Next, we will discuss about software tools that solve the problem.
\item Design and architecture (29/09/15 -- 23/11/15) \hfill \\
On the second phase, we are going to design the software architecture.
The architecture is the core of how software must behave, also to get an overview of system interaction.
Later, we are going to design a \gls{gui} of this system.
\item System implementation (7/12/15 -- 22/02/16) \hfill \\
This phase begins writing source codes based on designed architecture and technical specification.
Later on, we plan to connect each system component together.
\item Testing (19/02/16 -- 07/03/16) \hfill \\
This phase ensures that systems runs correctly with user and system requirements.
We will conduct testing phase on user acceptance.
Then, We will evaluate testing result.
If the result is satisfying, we will deploy the system to IC's server.
\end{enumerate}

\begin{landscape}
\begin{figure}
\centering
\caption{Project's schedule shown as Gantt chart}
\label{fig:project-schedule}
\includegraphics[scale=0.4]{res/project_plan}
\end{figure}
\end{landscape}