\chapter{Introduction}
\pagenumbering{arabic}

\section{Motivation}
\label{sec:motivation}

\gls{ic} is a faculty of \gls{kmitl} has about 25 employees, there are have two types of employees which are academic staff and adminstrative staff.

Their main objective is to provide educational services to students. 
Currently, they are having three major problems involving accumulating documents.
First, the organization takes countless hours to store and retrieve documents even though they have organized documents into categories.
Simply retrieving a document can take hours or even weeks. This problem hinders employee's productivity when they need to retrieve many documents. 
Second problem is losing track of documents, when the documents have to pass through many departments its often get lost along the way of workflow; A workflow is a series of repeatable steps performed in a sequential manner to complete a task.
% For instance, Mr.A wants video \enquote*{X} hosted on a website \enquote*{Y} to be taken down due to copyright violation.
% He has to go through Y's copyright claim workflow to satisfy his goal.
% First, Mr.A have to report the claim to Y.
% Y's administrator evaluates the claim.
% If the claim is valid, administrator takes down X.
% This task marks the end of Y's workflow that Mr.A has to go through.
% The workflow provides specification, execution and control of business processes \cite{Jablonski:1996:WMM}. 
% A person who hold responsibility of a document will execute actions based on their assigned roles.

Third problem is that some \gls{ic}'s workflows are so complicated that they are difficult to keep track of.
There are many actions and conditions to execute.
Many people from both inside and outside of the organization may be involves during workflow's execution.
Supplementary documents, or so called attachments, may need to be attached with original document in order to pass some certain guidelines.
As a result, \gls{ic} can't certainly identify who is currently involved and what actions they must do with that document.
They can not track attachments effectively causing delay in archiving and error-prone.

We are working on \MakeLowercase{\projTitle} for \gls{ic}, \kmitl, called Monkey Office. Because we want to provide a system to manage documents within \gls{ic}. Our proposed solution is to manage, store, and retrieve documents digitally using a computer software.
The employee will use less time to search and be less error-prone.
We will create a system that provide ability to search and to track documents.
When we have developed search and track documents, \gls{ic} can manage their own files and track documents to get increase organization's productivity. 

The similarities between \gls{dms} and file server is to provide a shared storage of computer files that can be accessed by other computers.
\gls{dms} aims to provide archive documents digitally with access controls while file server aims to store and retrieve files.
The differences between \gls{dms} and file servers are file to file relationship and predetermined workflow.
A file to file relationship is a built-in \gls{dms} feature that shows additional documents inside an organization.
Usually, a single document relates to other documents.
For example, a fire insurance contract refers to property insurance contract and fire policy.
Predetermined workflow is an already created workflow assigned to one or multiple documents.
\gls{dms} automatically assigns access rights to documents based on current workflow's task.
One of these two features suppose to have in order to be \gls{dms}.
Some \gls{dms} provide additional features such as version control or document assembly to satisfy organization's requirements.

\section{Objective}
\label{sec:objective}
% list of our primary objectives
Our primary objectives are
\begin{enumerate}
\item To design \gls{ic}'s workflow model\footnote{
	Although workflow is not our main focus, we work closely with other researchers on \enquote{web-based automatic form generated system} \cite{web-based-form}.
	Not only we share a similar objective, but also we have to link our system with them.
	We need their general workflow models to achieve this objective.
} associated with each type of documents.
\item To track and to report the current task during workflow's execution. To store and retrieve digital documents corresponding to outgoing or finished workflows from and to \gls{ic}'s server.
\end{enumerate}

\section{Terminology}
\label{sec:terminology}
We would like to clarify these two words---workflow and document---in terms of our project. %transition
% Explain what is a workflow
A \enquote{workflow} is a collection of repeatable tasks performed to achieve a goal \cite{Jablonski:1996:WMM}.
It can execute multiple tasks sequentially or simultaneously depending on how it is created.
For instance, there are four tasks namely \enquote*{P}, \enquote*{Q}, \enquote*{R}, and \enquote*{S}.
The workflow executes P first followed by Q.
Then, it executes R and S at the same time.
The execution of P and Q are sequential because Q have wait for P to finish first.
On the other hand, the execution of R and S are simultaneous because R and S execute in parallel.
In organizations, the goal of the document is get them approved, distributed, or archived.
Each task may perform with specific conditions.
A person, machine, group of persons or machines are responsible to carry out the task \cite{wfMangement}. 
% Explain what is a document
A \enquote{document} is a computer's executable file associates with each user's workflow.
For example, a user executes workflow \enquote*{X}.
One of the X's task requires user to upload a \gls{pdf} file as an input.
The task will produce a reply letter as an output.
There are two documents in X---the \gls{pdf} report and the reply letter.
% Explain how workflow and document are related
So, how workflow and document are related together?
A document can be an input or end-product of the task.
The task may require a certain document to execute.
Other documents may be produced by the task.


\section{Scope}
\label{sec:scope}
The similarities between \gls{dms} and file server is to provide a shared storage of computer files that can be accessed by other computers.
\gls{dms} aims to provide archive documents digitally with access controls while file server aims to store and retrieve files.
The differences between \gls{dms} and file servers are file to file relationship and predetermined workflow.
A file to file relationship is a built-in \gls{dms} feature that shows additional documents inside an organization.
Standalone organization's documents are rare.
Usually, they relate to other documents.
For example, a fire insurance contract refers to property insurance contract and fire policy.
Predetermined workflow is an already created workflow assigned to one or multiple documents.
\gls{dms} automatically assigns access rights to documents based on current workflow's task.
One of these two features suppose to have in order to be \gls{dms}.
Some \gls{dms} provide additional features such as version control or document assembly to satisfy organization's requirements.

We are going to develop a web-based application hosted on an \gls{ic}'s server.
Only people within \gls{ic} and granted external users can access the system.
Our primary focus is storing digital documents produced from tasks in the workflow.
The system has to manage workflows which are created by the staff of \gls{ic} and resulting documents.
Only documents are doing on the workflows, in the scope of this project.
This thesis conducts only on the following two types of documents.
% How these 7 types are related to form, resulting docs, attachments, ...
\begin{enumerate}
\item Form \hfill \\
A formatted document with blank fields that prompts user to fill in which generated the form by the admin on Web-based workflow management system.
Forms are created according to \gls{ic}'s work procedure and instruction.
Work procedure provides details on working procedures, departments, or persons who hold responsibilities for documents.
Work instruction indicates employee's roles and responsibilities.
\gls{ic} defines non-filled form as \enquote*{Form} and already filled form as \enquote*{Record}.
In this project conduct with three sub-types forms.
\begin{enumerate}
\item Absence form \hfill \\
For \gls{ic}'s employee and students to take days off due to personal or business reasons.
\item Student internship form \hfill \\
A form for \gls{ic} students to take an internship.
\item Conference outside \gls{kmitl} form \hfill \\
For lecturers to request going to a conference outside \gls{kmitl}.
\end{enumerate}
Because all of these forms are in electronic format, we have to complete the tasks of workflow then it can be convert them into the document file.
% Meaning that a system displays the form on a computer screen waiting for a user to fill in required blank fields.
\item Physical document \hfill \\
Documents received from outside or physical document from IC's departments in order to proceed some tasks in a workflow.
If received document is in paper form, it has to be scanned by a scanner first.
The task includes these documents as attachments.
\end{enumerate}