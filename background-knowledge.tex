\chapter{Background Knowledge}

\section{A brief overview of \gls{dms}}
In the 1980s, printers, scanners, and household computers started to gain popularity.
Organizations start to take actions on managing their information records and assets seriously.
At that time, Document Image Processing (DIP) systems are the only available software to satisfy their needs.
DIP is the electronic version of filing cabinet where documents need to be scanned, indexed, and store in the system \cite{1_adam_2008}.
DIP can also analyse figures, texts, and handwriting \cite{akram2010document} using various image processing techniques.
\begin{wrapfigure}{l}{0.5\textwidth}
	\centering
	\includegraphics[scale=0.7]{res/bg-knowledge/edms-components.png}
	\caption{Components of EDMS \citefigure{1_adam_2008}}
	\label{fig:edms-components}
\end{wrapfigure}
Later on in the 1990s, Electronics Document Management System (EDMS) is developed targeting large enterprises with high volume of documents.
It is the improved version of DIP with workflow functionality.
Workflow functionality enables organizations to passed around scanned document throughout the organization to designated employee.
EDMS also has its own document repository allowing documents to be indexed and tracked using version control.
There are various sub-types of EDMS such as Electronic Record Management Systems (ERMS) deals mainly with record keeping.
Enterprise Content Management (ECM), suites of applications, deals with document and record management.
Nowadays, EDMS acronym is shortened to DMS.
Figure \ref{fig:edms-components} shows a basic components of DMS.
Most DMS applications have these components implemented within.
\begin{description}
	\item[Document Repository] \hfill \\
	A place where indexed documents are stored.
	Typically on a hard disk of a network server.
	\item[Integration with desktop applications] \hfill \\
	Allowing user to save documents straight to application when document is created.
	It is usually a 3rd party add-on embedded in popular office applications such as Microsoft Office.
	\item[Check-in and check-out] \hfill \\
	This feature controls who is allowed to make changes or to read documents.
	Basically, it is a user permission control system.
	\item[Versioning] \hfill \\
	Keeping track of changes by assigning a version number to a document.
	The number is incremented after the document passes major revisions.
	User can access the previous versions of the document.
	\item[Auditing] \hfill \\
	Track who changes document, when, and where.
	Only authorized user can read these information.
	\item[Security] \hfill \\
	Controls how documents should be stored in the server to prevent hackers from attacking the system.
	\item[Classification and indexing] \hfill \\
	Metadata and tags provide more information to documents.
	It helps to search and retrieve documents easier.
	\item[Search and Retrieval] \hfill \\
	Allow user to retrieve documents according to keywords.
	Keywords can be a metadata or contents within a document.
	A system may offer advance search criteria by looking for individual fields and combine with other fields using basic logical operations (AND, OR, NOT). 
\end{description}

\section{What is Metadata}
Metadata literally means \enquote{data about data} \cite[p.~1]{baca_2008}.
Brackett \cite[p.149]{brackett_2000} defined metadata in terms of organization as \enquote{Any data about organization's resource}.
There is no clear definition on what metadata is because this term is used differently in different communities.
For librarians, it refers to information in the library catalog that help users to find the right book in the library.
For search engines, it means descriptions of web page's contents and keywords used to rank relevant websites.
For others, it may refer to a descriptive information of resources in human readable format.
Whatever metadata refers to, they share the following similar usage.
\begin{enumerate}
	\item To identify resources.
	\item To distinguish or bundle similar resources.
	\item To find resources based on search criteria.
\end{enumerate}
There are 3 types of metadata \cite{hodge_2001}.
\begin{description}
	\item[Descriptive metadata] give a description of resources for discovery and identification.
	\item[Structural metadata] describes how component's objects are organized.
	\item[Administrative metadata] indicates information about how resource suppose to be managed.
\end{description}

\begin{figure}[h]
	\centering
	\includegraphics[scale=0.8]{res/bg-knowledge/data-and-metadata}
	\caption{data and metadata \citefigure{hay_2006}}
	\label{fig:data-metadata}
\end{figure}
Figure \ref{fig:data-metadata} provides simple model examples of how metadata represents real-world things.
Each column shows different examples.
The third row is a descriptive metadata.
Julia Robert is the customer's name and 10/28/67 is her birthdate.
These 2 metadata refer to only this \enquote{Julia Roberts} \includegraphics[scale=0.3]{res/bg-knowledge/metadata-julia} and not any other Julia Roberts. 
Same with a second example, A branch in \enquote{111 Wall Street} with a branch manager \enquote{Sam Sneed} refer to this \includegraphics[scale=0.3]{res/bg-knowledge/metadata-wallstreet} branch only.
The next row up is a structured metadata.
It shows a higher abstraction concept to describe general real-world things.
Julia Roberts can be generalized as a \enquote{Customer} entity with \enquote{Name} and \enquote{Birthdate} as its attributes.
A checking account in the third example can be generalized as a table called \enquote{CHECKING\_ACCOUNT} with \enquote{Account\_number} and \enquote{Monthly\_charge} as its columns.

\section{NoSQL}
\gls{nosql} is a database that does not use \gls{sql}.
It refers to any database that does not follow the traditional \gls{rdbms} model.
\gls{sql} was designed to be a query language for relational databases, and they are usually table-based.
Records are stored in rows and columns represent fields.
On the other hand, \gls{nosql} allow to define fields while creating a record.
Nested values are common in \gls{nosql} databases because \gls{nosql} is aggregate oriented.
Hashes and arrays and objects, and then nest more objects and arrays and hashes within those.

The main characteristic separating \gls{nosql} databases from relational \gls{sql} databases is that they do not use query languages derived from \gls{sql}. 
The following list shows common features of \gls{nosql} \cite[p.~12 - 16]{nosql-for-dummies}.
\begin{description}
	\item[Schema agnostic] \hfill \\
	\gls{nosql} databases doesn't require schema to be defined explicitly.
	Any type of records can be stored without having to know how databases store them internally.
	\item[Nonrelational] \hfill \\
	A relational database needs relations to describe how tables relate to each other.
	Unlike \gls{rdbms}, \gls{nosql} databases don't have any relation concept.
	They don't store how each record relate to each other but rather graph database with key-value pairs.
	\item[Highly distributable] \hfill \\
	A single server may not be able to handle all data requests in time.
	Instead of dedicating database on the single server, many servers needed to process queries in parallel.
	Storing data across multiple servers in relational databases is a challenging task.
	\gls{nosql} databases can handle distributed queries as long as connected machines are fast enough to talk to each other.
\end{description}

There are many \gls{nosql} storage types available to model the content.
For example, \enquote{Column-oriented database} stores data as columns instead of rows in \gls{rdbms}.
\enquote{Graph store} represents data as nodes and relationships as edges in a graph.
This report will focus on \enquote{Document Store} storage type.
\gls{nosql} databases with \enquote{Document Store} storage type organized data as a hierarchical model with parent-child relationships.
The topmost node in Figure \ref{arch-node} called a root node.
This node is required as a starting point for a record and the model must have only one root node.
\begin{figure}[h]
	\centering
	\includegraphics[scale=0.48]{res/bg-knowledge/nosql-hierarchical}
	\caption{The hierarchical organized into a set of parent-child nodes \citefigure{nosql-for-mere-mortals}}
	\label{arch-node}
\end{figure}
Directed Edges connecting between two nodes represents relations.
A node where an arrow points called a parent node.
A node where an arrow is pointed called a child node.
The parent node can many child nodes and each child node can have many parent nodes.

For example, a simple customer's receipt with orders and billing addresses from figure \ref{nosql-aggregate-uml} can be modelled to hierarchical parent-child structure in figure \ref{nosql-aggregate-uml-realized}. 
Each node is an object in itself.
\gls{nosql} database can locate any node within the graph according to any specific queries.
\begin{figure}[h]
	\centering
	\includegraphics[scale=0.55]{res/bg-knowledge/nosql-nosql-agregate-uml}
	\caption{An aggregate data model in UML notation \citefigure{sadalage_fowler_2013}}
	\label{nosql-aggregate-uml}
\end{figure}
\begin{figure}[h]
	\centering
	\includegraphics[scale=0.5]{res/bg-knowledge/nosql-uml-realized}
	\caption{A hierarchical structure representation from figure \ref{nosql-aggregate-uml}}
	\label{nosql-aggregate-uml-realized}
\end{figure}

Most databases use \gls{xml}, \gls{json}, \gls{bson}, or \gls{yaml} to manipulate data.
Client have to use REST API provided by the database to query the data.
\gls{rest} is a network-based software architectural style of the World Wide Web \cite{doglio, masse_2012}.
It takes care interactions between client and server.
Client program communicates with the database using \gls{http} through exposed \glspl{api}.

Table \ref{tbl:doctype-storage-feature} summarize key features of \gls{nosql} database with \enquote{Document Store} storage type.
\begin{table}[h]
	\centering
	\begin{tabular}{l c}
		\hline
		Feature                             &      Document Store \\
		\hline
		Table-like schema support (columns) &      No \\
		Complete update/fetch               &      Yes \\
		Partial update/fetch                &      Yes \\
		Query/Filter on	value               &      Yes \\
		Aggregates across rows              &      No \\
		Relationships between entities      &      No \\
		Cross-entity view support           &      Yes \\
		Batch fetch                         &      Yes \\
		Batch update                        &      Yes \\
		\hline
	\end{tabular}
	\caption{Sumarization of available features for \enquote{Document Store} storage type \citefigure{vaish_2013} \cite{vaish_2013}}
	\label{tbl:doctype-storage-feature}
\end{table}

\glsreset{bpmn}
\section{\gls{bpmn}}
\gls{bpmn} is a graphical notation that depicts the steps in business process \cite{bpmn_omg}.
% What
The goal is to represent business processes using standard graphical notation.
% Who
\gls{bpmn} targets business users and process implementers who need a standard model to communicate their business process.
% Why
Business users create, manage, and monitor processes while process implementers turn processes into a physical implementation. 
\gls{bpmn} doesn't focus on why, when, and how a process is performed.
But rather what processes, which are the steps to achieve that process, and who should do them.
With \gls{bpmn}, organizations can understand, improve, and control their business processes.
Table \ref{tbl:sum-bpmn-symbol} summarizes \gls{bpmn} symbols that appear in this report along with their descriptions\footnote{Please refer to \url{http://www.omg.org/cgi-bin/doc?formal/11-01-03.pdf} (page 28 - 41) for more details on all available \gls{bpmn} symbols and its detailed descriptions.}.
%TODO UPDATE NEW SYMBOLS AS IT APPEARS IN THIS REPORT
\begin{longtable}{C{0.2\textwidth} L{0.45\textwidth} C{0.3\textwidth}}
	\caption{Summary of \gls{bpmn} symbols \citefigure{bpmn_manual_omg} \cite{bpmn_manual_omg}}
	\label{tbl:sum-bpmn-symbol} \\
	
	% Setting up header-----------------------------
	\hline
	Element & Description & Notation \\
	\hline
	\endhead
	
	\hline \multicolumn{3}{r}{{Continued on next page}} \\ \hline
	\endfoot
	
	\hline \hline
	\endlastfoot
	%-----------------------------------------------
	
	Start Event & 
	Indicates where a process begins &
	\bpmnfig[scale=0.015]{\bpmnSymRepo start-event-none} \\
	
	Intermediate Event &
	Intermediate Event represents a status reached in a process, and it is modelled explicitly.  &
	\bpmnfig[scale=0.02]{\bpmnSymRepo start-event-intermediate} \\
	
	End Event & 
	Indicates where a process ends &
	\bpmnfig[scale=0.015]{\bpmnSymRepo end-event-none} \\
	
	End Event Terminate &
	Terminate all running activities in a process immediately without any error handling. &
	\bpmnfig[scale=0.015]{\bpmnSymRepo end-event-terminate} \\
	
	Message &
	A Message represents information communicating between two participants. &
	See two rows below. \\
	
	Start Message Event &
	An event triggers a process to start when a Message arrives from a participant. &
	\bpmnfig[scale=0.025]{\bpmnSymRepo start-event-message} \\
	
	End Message Event &
	A message received from a participant concludes the end of a process. &
	\bpmnfig[scale=0.025]{\bpmnSymRepo end-event-message} \\
	
	Activity &
	An activity performs work within a business process.
	It can be single or compound.
	It is an executable element of \gls{bpmn}. &
	See three rows below. \\
	
	Task & 
	A Task is a unit of work performs in a process.
	It has a task name inside a rounded rectangle notation.
	The task is used when process can not be broken down any further.
	It has no internal parts, representing a single action. &
	\bpmnfig[scale=0.03]{\bpmnSymRepo task} \\	
	
	User Task &
	A subtype of Task in which human is the one who perform a task with the help of a software application. &
	\bpmnfig[scale=0.035]{\bpmnSymRepo user-task} \\	
	
	Sub-Process &
	A Sub-Process composes of a single activity or compound activity.
	It refers to a process that can be broken down to a finer detail.
	The collapsed Sub-Process hides activity details from the diagram indicated by a plus sign.
	The expanded one shows activities inside the diagram indicated by a minus sign. &
	\bpmnfig[scale=0.03]{\bpmnSymRepo subprocess-collapsed}
	Collapsed Sub-Process
	\bpmnfig[scale=0.03]{\bpmnSymRepo subprocess-expanded}
	Expanded Sub-Process \\	
	
	Activity Looping &
	Activity Looping loops itself continuously until its boolean condition becomes \textit{true}.
	It evaluates condition every time before/after it performed for every loop iteration. &
	\bpmnfig[scale=0.04]{\bpmnSymRepo task-loop} \\	
		
	Normal Flow &
	Shows activity order performed in a process &
	\bpmnfig[scale=0.02, angle=-45]{\bpmnSymRepo connection} \\
	
	Conditional Flow &
	It will perform the outgoing sequence flow if and only if conditional expression evaluates to \textit{true}. &
	\bpmnfig[scale=0.02, angle=-45]{\bpmnSymRepo conditional-flow} \\

	Default Flow &
	Execute this flow if all other outgoing conditional expression evaluate to \textit{false}. &
	\bpmnfig[scale=0.02, angle=-45]{\bpmnSymRepo default-flow} \\

	Message Flow &
	Indicates a Message sent between two participants. &
	\bpmnfig[scale=0.025, angle=-45]{\bpmnSymRepo message-flow} \\

	Gateway & 
	Controls branching, merging, forking, and joining multiple sequence flow paths. 
	& See four rows below. \\
	
	Exclusive Gateway &
	Creates multiple alternative paths in a process.
	If one of the conditional expression evaluates to \textit{true}, it executes that path without evaluating others.
	It has the same functionality as conditional flow.
	The difference is that Exclusive Gateway allows multiple branching from the same node. &
	
	\begin{center}
	\begin{minipage}{0.1\textwidth}
		\bpmnfig[scale=0.025]{\bpmnSymRepo gateway-none}
	\end{minipage}
	or
	\begin{minipage}{0.1\textwidth}
		\bpmnfig[scale=0.025]{\bpmnSymRepo gateway-xor}
	\end{minipage}
	\end{center} \\
	
	Event-based Gateway &
	Evaluates which event occurs first, not which condition is met.
	If that event occurred, go to that outgoing flow and discard other conditions. &
	\bpmnfig[scale=0.025]{\bpmnSymRepo gateway-eventbased} \\
	
	Parallel Gateway &
	Parallel gateway is used to create and to combine parallel flows.
	It represents multiple concurrent activities performing at the same time.
	It doesn't evaluate any condition or event before execution.
	Parallel gateway will wait for all incoming flows to finish before proceeding to the next outgoing flow. &
	\bpmnfig[scale=0.025]{\bpmnSymRepo gateway-parallel} \\	
	
	Inclusive Gateway &
	Evaluates all conditional expression unlike the exclusive gateway.
	It is used to create alternative parallel paths.
	If there are multiple conditional expressions evaluated to \textit{true}, those flows will be executed independently in parallel. &
	\bpmnfig[scale=0.025]{\bpmnSymRepo gateway-or} \\		
	
	Data Object &
	Provide information about what activity requires to perform and/or what they produce. 
	Data Object can also refer to a collection of Data Objects.
	Data Input have a white arrow on the top corner.
	It is placed before activity starts indicating what that activity requires.
	Data Output have a shaded arrow on the the corner.
	It is placed after activity finished indicating what activity produces. &
	\bpmnfig[scale=0.025]{\bpmnSymRepo data-object}
	Data Object
	\bpmnfig[scale=0.025]{\bpmnSymRepo data-input}
	Data Input
	\bpmnfig[scale=0.025]{\bpmnSymRepo data-output}
	Data Output \\
	
	Text Annotation &
	Text Annotation provides additional information to a \gls{bpmn} diagram reader.
	Its notation composes of a opening square bracket with a dashed line attached on the left.
	The dashed line can connect to any \gls{bpmn} element.
	The bracket contains a descriptive text on the right.
	It scales vertically to cover multiple lines of text.
	&
	\bpmnfig[scale=0.025]{\bpmnSymRepo text-annotation} \\
	
	Pool &
	A Pool represents a collaboration between participants in a process.
	It can be partitioned to multiple smaller Pools called Lane.
	Each Lane assigns to one participant.
	The Participant is a person, machine, group of persons or machines responsible for the process execution within the Pool.
	The Pool is a container of activities.
	All activities inside the Pool can cross boundaries to any Lane but not outside the Pool.
	There are two types of Pool---horizontal and vertical.
	Both of them has the same functionality.
	The difference is their alignment.
	A horizontal one expands horizontally while a vertical one expands vertically.
	
	&
	\bpmnfig[scale=0.03]{\bpmnSymRepo participant}
	Horizontal Pool
	\bpmnfig[scale=0.03]{\bpmnSymRepo collaboration}
	Horizontal Pool with Lanes
	\bpmnfig[scale=0.03, angle=-90]{\bpmnSymRepo participant}
	Vertical Pool
	\bpmnfig[scale=0.03, angle=-90]{\bpmnSymRepo collaboration}
	Vertical Pool with Lanes. \\
	
	\hline

\end{longtable}

\newpage

% Teach readers how to read a BPMN diagram by example.
\begin{figure}[t]
	\centering
	\includegraphics[scale=0.55]{res/bg-knowledge/bpmn-example}
	\caption{Shipment Process of a hardware retailer \citefigure{bpmn_manual_example_omg} \cite{bpmn_manual_example_omg}.}
	\label{fig:bpmn-example}
\end{figure}

Figure \ref{fig:bpmn-example} provided an example of a \gls{bpmn} diagram.
Note that all gateways come in pairs.
Exclusive Gateway have to indicate where branching starts and merges.
Parallel Gateway and Inclusive Gateway need to synchronize flows because branched tasks are performed in parallel.
Synchronization is to ensure that all branched tasks are completed before going to the next flow.

The figure shows steps of hardware retailer's shipment process of shipping goods to customers.
There are three participants involved in the hardware retailer---a warehouse worker, a clerk, and a logistics manager.
A Start Event (\includegraphics[scale=0.0035]{\bpmnSymRepo start-event-none}) marks the beginning of a process.
An End Event (\includegraphics[scale=0.0035]{\bpmnSymRepo end-event-none}) marks the end of the process.
Arrows navigates the flow to each node in the diagram.

\enquote{Goods to ship} is a trigger of a process.
A Parallel Gateway (\includegraphics[scale=0.005]{\bpmnSymRepo gateway-parallel}) with outgoing two flows indicates that task \enquote{Decide if normal post or speical shipment} and \enquote{Package goods} have to be done in parallel.
While clerk is deciding if goods is a normal postal or special shipment, the warehouse worker can start packaging the goods.
Next clerk's task is to decide a \enquote{Mode of delivery} indicated by an Exclusive Gateway (\includegraphics[scale=0.005]{\bpmnSymRepo gateway-none}).
The gateway is not responsible for clerk's decision, but rather the task before it.
If the clerk decides that goods are a special carrier, the clerk will perform \enquote{Request quotes from carriers} and \enquote{Assign a carrier \& prepare paperwork} task.
A special carrier includes a carrier service denoted by a Text Annotation (\includegraphics[scale=0.005]{\bpmnSymRepo text-annotation}).
If the clerk decides that goods are normal post, the clerk will perform \enquote{Check if extra insurance is necessary} task.
While the clerk is filling in a post label, if goods requires an extra insurance, the logistics manager will take out extra insurance.
An Inclusive Gateway (\includegraphics[scale=0.005]{\bpmnSymRepo gateway-or}) ensures that logistic manager will deal with goods if they need the extra insurance.
Finally, the warehouse worker will add paperwork and move package to a pick area.
All goods are ready to ship to customers.