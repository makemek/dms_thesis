\chapter{Literature Review}
This chapter introduces some existing document management system.
Section \ref{dms-features} gives lists of required features of \dms.
Section \ref{relate-works} explores some existing document management applications.

\section{Integrated features} \label{dms-features}
The features of \dms contains the following
\begin{description}
\item[Search file-reference]
User can search documents and their dependencies based on these keywords
\begin{enumerate}
\item Title
\item Description
\item Author's name
\item Date
\item Category
\end{enumerate}

\item[Upload/download file]
\item[Print documents]
\item[Generate ID and version of each document]
\item[Provide document's status]
\end{description}

\section{Related Works} \label{relate-works}
We pick 9 existing document management applications to investigate.
Each application has its interesting functionalities and characteristics.
We focus mainly on these 2 features, access control and document retention.
\begin{description}
\item[Alfresco One] \hfill \\
Alfresco One allows organizations to manage any type of content from small office documents, photographs, to large video files.
The system allows users to choose how to access their content --- via the Web, desktop or email --- while the server enforces access controls and security.

\item[Dockmee] \hfill \\
Dokmee has multiple editions targeted at companies of all sizes.
It can run in a Windows-based Intranet network, as a Web-hosted system or as a software-as-a-service model.
User can organize files into folders and can store unlimited number of files.


\end{description}