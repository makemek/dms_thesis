\chapter{Conclusion}

%problem description
In conclusion, this thesis is about the Opensources-based Document Management System (DMS), which is a system proposed for the staffs and the lecturers in International College(IC), KMITL, to work together efficiently. Problems occur at the present, managing documentation's operation in the organization, is mostly manual and uses physical document to work on. When people keep their files in their own places. The problems include as following,
\begin{enumerate}
\item The waste of time when a person needs some information from other people and has to meet them directly to get the information he want.
\item The waste of time and manpower when people have to search for specific documents from many unrelated documents.
\item The difficulty of document tracking, in which people have to know others' progress and where they keep their works.
\item The complexity of workflows, in which people have to work together, follow specific sequences as well as deal with different places and different type of documents.
\end{enumerate} 


Monkey Office is the system to manage all documents in IC KMITL. It is created for solving all above problems. In the system, there are three parts which are the following,
\begin{enumerate}
	\item QUALITY ASSURANCE SYSTEM(QA) is the systematic review of educational programmes to ensure that acceptable standards of education, scholarships and infrastructure are being maintained. (UNESCO) %Reference%

	\item WEB-BASE WORKFLOW MANAGEMENT SYSTEM(WF) is the system that help people can track the documents as they follow a workflow or, even better, a system which could help automate entire workflows. 

	\item OPENSOURCES-BASED DOCUMENT MANAGEMENT SYSTEM(DMS) is helping to manage, store and retrieve documents. 
\end{enumerate}

This thesis is the last one, DMS. IC's staffs and lecturers can use it to manage, store, and retrieve the documents conveniently. DMS solves the problems by being a center of information. It lets people share their files. Moreover, users can search the documents: 1) document name, 2) document's author, and 3) created date.
Furthermore, each public file, they can see all above details and the following details which are related works, document's version, and some description of the document. %also they can execute them from the permission access control. 


Further development, DMS will available for other users and students can access to their course materials and other information online such as internship, announcement, and time table. Besides, the system can analyze all documents, then convert it to visual statistic. For example, how many internship documents are approved in 2016?.