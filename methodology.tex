\chapter{Methodology}
\section{Introduction}
The overall purposes of this study is to improve efficiencies of \gls{ic}'s digital document management strategies. 
The study involves investigating internal \gls{ic}'s working procedures and regulations.
How \gls{ic} execute strategies on their digital document assets.
What are advantages and disadvantages of their strategies.
Gaining such information is valuable to this study because plausible solutions could be proposed and implemented. 
Solutions that could further improve flows of documents inside \gls{ic}.
This chapter presents reasonable research methods to address the research questions.
Conceptual design will also be introduced to provide an abstract model of this experiment.

\section{Research Strategy}
Section \ref{sec:motivation} introduced digital document problems occurred inside \gls{ic}.
These problems were formulated to research questions suitable for this study.
The study's goal is to create a system to verify the hypotheses posed by three research questions:
\begin{enumerate}
	\item What are the critical factors contribute to extreme delay in searching and retrieving documents even though IC have already organized them?
	\item How to store and retrieve digital \gls{ic}'s documents so that authorized users can access within \gls{ic}, \gls{kmitl} precinct?
	\item How to track \gls{ic}'s document going through each task specified by \gls{ic}'s document workflow?
\end{enumerate}
This study implements qualitative research method with three research strategies: interview, case study, and software architectural model.
The following sections discuss these three strategies in more detail.

\subsection{Qualitative Research}
Qualitative research is a systematic process of inquiring or investigating to inform and to decide a course of action \cite{merriam2015qualitative}.
Its main purpose is to establish a basis for decision making, \enquote{to make judgement about the program, improve program effectiveness, and/or inform decisions about future programming} \cite{patton2005qualitative}.

\subsubsection{Interview}
\citeauthor{gall7j} (\citeyear{gall7j}) states that interview is the spontaneous generation of questions in a natural interaction, typically one that occurs as part of ongoing participant observation fieldwork.
\citeauthor{brady2011craft} (\citeyear{brady2011craft}) points out that what interviewer and salesman have in common is potential customers whom one could hold their attention to talk.
Getting an interview means making an appointment to see the subject, identifying questions related to the research topic, and showing on time for the interview.
The purpose of interview is to gain information from interviewee by having interviewer asking questions.
Researchers can gain useful insights from the subject who is expertise in one's field.
In this study, \gls{ic} staffs are the subject of this study divided into two focus groups:
\begin{description}
	\item [Administrative staffs] as a primary focus group because they are responsible for keeping records of all \gls{ic}'s documents.
	Transfer documents around the organization according to \gls{ic}'s workflow specifications.
	Managing day-to-day operations within the organization.
	They also provide academic advices and guidances to \gls{ic} undergraduate and postgraduate students.
	\item [Academic staffs] as a secondary focus group because they are not involve in keeping and organizing documents, but rather implementers.
	Some of them have privilege to issue, review, or approve documents.
	In order to know more about their procedures, they are selected as a sample for this study.
\end{description}

\subsubsection{Case Study}
\citeauthor{merriam1988case} (\citeyear{merriam1988case}) defines case study as an intensive study of a single unit for the purpose of understanding a larger class of similar units.
Unit refers to an observed sample at a discrete point in time.
Each unit comprises of cases built from variables upon observations.
For example, an analysis of worldwide smartphone sales in fourth quarter of 2011 \cite{goasduff2012gartner}.
Samples are mobile devices.
Units are worldwide mobile sales to end users observed in fourth quarter of 2010 and 2011.
Cases are sales by vendor and sales by operating system.
Variables are total number of sales (thousands of units) and market share percentage.
Figure \ref{tbl:ex-case-study-var} shows \citeauthor{goasduff2012gartner}'s case study \cite{goasduff2012gartner} in hierarchy based on \citeauthor{merriam1988case}'s definition.
A unit (sales in fourth quarter) are constructed from two sub units based on years---2010 and 2011.
Sub units in figure \ref{tbl:ex-case-study-var} are split to another column to preserve space.

\begin{figure}[h]
	\caption{Samples, units, cases, and variables for \citeauthor{goasduff2012gartner}'s case study \cite{goasduff2012gartner}}
	\label{tbl:ex-case-study-var}
\begin{tabular}{ll}
	\begin{minipage}{8cm}\dirtree{%
		.1 Mobile Devices.
			.2 Sales in Fourth Quarter.	
				.3 2010.
					.4 By Vendor.
						.5 Total Sales (Thousands of Units).
						.5 Market Share (\%).
					.4 By Operating System.
						.5 Total Sales (Thousands of Units).
						.5 Market Share (\%).
		}
	\end{minipage}
	&
	\begin{minipage}{8cm}\dirtree{%
		.1 Mobile Devices.
			.2 Sales in Fourth Quarter.	
				.3 2011.
					.4 By Vendor.
						.5 Total Sales (Thousands of Units).
						.5 Market Share (\%).
					.4 By Operating System.
						.5 Total Sales (Thousands of Units).
						.5 Market Share (\%).
		}
	\end{minipage}
\end{tabular}
\end{figure}

Conducting case studies based on \citeauthor{merriam1988case}'s definition can yield concrete results.
\citeauthor{merriam1988case}'s definition offers an advantage to study the same case in different point of time.
It also allows variables to be unquantifiable as \citeauthor{merriam1988case} states that variables are built from observation.
This goes along with the qualitative research method that relies on inquiring and investigating to decide a course of action.

This study utilizes \citeauthor{merriam1988case}'s case study model to construct case studies as shown in figure \ref{fig:our-case-study-method}.
Documents are a sample.
\gls{ic}'s documents observed in 2015 is a unit.
There are three cases;
\begin{enumerate*}
	\item storing digital documents;
	\item retrieving digital documents;
	\item tracking digital documents.
\end{enumerate*}
First and second unit comes from the second research question.
Thrid unit comes from the third research question.
Each case comprises of four variables;
\begin{enumerate*}
	\item absence form;
	\item student internship form;
	\item conference outside \gls{kmitl} form;
	\item external document.
\end{enumerate*}
They come from the scope stated in section \ref{sec:scope}.

\begin{figure}[h!]
	\centering
	\caption{Samples, units, cases, and variables for case study of this research}
	\label{fig:our-case-study-method}
	\begin{minipage}{8cm}\dirtree{%
		.1 Documents.
		.2 \gls{ic}'s Documents Observed in 2015.
		.3 Storing Digital Documents.
		.4 Absence Form.
		.4 Student Internship Form.
		.4 Conference Outside \gls{kmitl} Form.
		.4 External Digital Documents.
		.3 Retrieving Digital Documents.
		.4 Absence Form.
		.4 Student Internship Form.
		.4 Conference Outside \gls{kmitl} Form.
		.4 External Digital Documents.
		.3 Tracking Digital Documents.
		.4 Absence Form.
		.4 Student Internship Form.
		.4 Conference Outside \gls{kmitl} Form.
		.4 External Digital Documents.
		}
	\end{minipage}
\end{figure}

% ***Note to self***
% Refer to https://www.cs.cmu.edu/~Compose/ftp/shaw-fin-etaps.pdf
% Architectual model are the research results of the experiment -> split to another chapter
% Implementation come from result validation -> split to another chapter

\section{Hypothesises}
\subsection{Extreme Delay in Searching and Retrieving Documents}
% Lack of employee?
% The way they organize files hinders retriving and searching eg. files spread all over the place, No standard folder structure and naming convention, everyone stores document locally not centralize
% They have something more important to do. As long as it is not urgent, they postpone it.

\subsection{Storing and Retrieving Documents within IC}
% **Store and retrieve***
% all digital original documents copy resides in the same server. If anybody need it, they can get it from one place.
% To get a document -> query database -> get the path to file and download it

% ***Acessing within IC***
% A close-network centralized server
% Authorization with account issued only by IC

\subsection{Tracking Documents}
% have a status flag. Some sort of extra metadata indicates what task document is going though at the momment
% - attach in file naming ex. approved_xxxx.doc, reject_xxx.doc, reviewing_xxx.doc not effective it make stuff's work even more tedious.
% - or require a database (a centralized place) to store such metadata.