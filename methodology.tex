\chapter{Methodology}
\section{Introduction}
The overall purposes of this study is to improve efficiencies of \gls{ic}'s digital document management strategies. 
The study involves investigating internal \gls{ic}'s working procedures and regulations.
How \gls{ic} execute strategies on their digital document assets.
What are advantages and disadvantages of their strategies.
Gaining such information is valuable to this study because plausible solutions could be proposed and implemented. 
Solutions that could further improve flows of documents inside \gls{ic}.
This chapter presents reasonable research methods to address the research questions.
Conceptual design will also be introduced to provide an abstract model of this experiment.

\section{Research Strategy}
Section \ref{sec:motivation} introduced digital document problems occurred inside \gls{ic}.
These problems were formulated to research questions suitable for this study.
The study's goal is to create a system to verify the hypotheses posed by four research questions:
\begin{enumerate}
	\item What are the critical factors contribute to extreme delay in searching and retrieving documents even though IC have already organized them?
	\item How to store and retrieve digital \gls{ic}'s documents so that authorized users can access within \gls{ic}, \gls{kmitl} precinct?
	\item How to track \gls{ic}'s document going through each task specified by \gls{ic}'s document workflow?
	\item How to model \gls{ic}'s workflows so that it can be executed automatically by a computer?
\end{enumerate}
This study implements qualitative research method with three research strategies: interview, case study, and software architectural model.
The following sections discuss these three strategies in more detail.

\subsection{Qualitative Research}
Qualitative research is a systematic process of inquiring or investigating to inform and to decide a course of action \cite{merriam2015qualitative}.
Its main purpose is to establish a basis for decision making, \enquote{to make judgement about the program, improve program effectiveness, and/or inform decisions about future programming} \cite{patton2005qualitative}.

\subsubsection{Interview}
\citeauthor{gall7j} (\citeyear{gall7j}) states that interview is the spontaneous generation of questions in a natural interaction, typically one that occurs as part of ongoing participant observation fieldwork.
\citeauthor{brady2011craft} (\citeyear{brady2011craft}) points out that what interviewer and salesman have in common is potential customers whom one could hold their attention to talk.
Getting an interview means making an appointment to see the subject, identifying questions related to the research topic, and showing on time for the interview.
The purpose of interview is to gain information from interviewee by having interviewer asking questions.
Researchers can gain useful insights from the subject who is expertise in one's field.
In this study, \gls{ic} staffs are the subject of this study divided into two focus groups:
\begin{description}
	\item [Administrative staffs] as a primary focus group because they are responsible for keeping records of all \gls{ic}'s documents.
	Transfer documents around the organization according to \gls{ic}'s workflow specifications.
	Managing day-to-day operations within the organization.
	They also provide academic advices and guidances to \gls{ic} undergraduate and postgraduate students.
	\item [Academic staffs] as a secondary focus group because they are not involve in keeping and organizing documents, but rather implementers.
	Some of them have privilege to issue, review, or approve documents.
	In order to know more about their procedures, they are selected as a sample for this study.
\end{description}

\subsubsection{Case Study}

\subsubsection{Software Architectural Model}

\section{Extreme Delay in Searching and Retrieving Documents}

\section{Storing and Retrieving Documents}

\section{Tracking Documents}

\section{Workflow Modelling}
