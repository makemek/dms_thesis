\chapter{Software Design}

\section{Use Case}
\begin{figure}[h]
	\includegraphics[scale=0.63]{res/Methodology/usecase_diagram}
	\caption{A usecase diagram for Monkey Office}
	\label{fig:usecase-diagram}
\end{figure}
Figure \ref{fig:usecase-diagram} captures system functionalities and requirements represented as \gls{uml} use case diagram.
Table \ref{tbl:actor-description} describes who these type of users are.
\begin{table}
	\centering
	\caption{Type of user and description}
	\label{tbl:actor-description}
	\begin{tabular}{|l|L{7.5cm}|}
		\hline
		Actor & Description \\
		\hline
		Dean & A headmaster of \gls{ic}. \\
		Deputy dean & Dean's assistant who could also be a dean's representative. \\
		Assistant of deputy dean & Deputy dean's assistant who deals with documents that is not critical to \gls{ic}. \\
		Staff & \gls{ic}'s employee in academic department (refer to bottom-most hierarchy in figure \ref{ic-org-sturcture}) \\
		User & A regular user who have an account provided by an administrator to access the system. \\
		\hline
	\end{tabular}
\end{table}

\section{Storing Digital Documents}
Before storing documents, there are two things to concern.
Firstly, what metadata should be stored alongside the document so that the system can access to the digital file including its attachments.
Attachments are documents.
They can be official or unofficial documents.
Both official and unofficial documents are the same except for one characteristic.
Official documents has a customized identification code as mentioned in table \ref{tbl-doc-subtype}.
Unofficial documents has only regular identification number starting at one and increment by one for each document.
Therefore, the parent document have to store a list of attachments.
That is a list referring to itself.
A class diagram \ref{fig-doc-template} shows 
%\begin{figure*}
%	\label{fig-doc-template}
%	\caption{text}
	%\includegraphics[scale=0.5]{res/Methodology/DocumentTemplating}
%\end{figure*}

Secondly, where digital files should reside in the the server.


\section{Retrieving Digital Documents}

\section{Tracking Digital Documents}