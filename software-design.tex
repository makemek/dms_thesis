\chapter{Software Design}

\section{Use Case}
\begin{figure}[h]
	\centering
	\includegraphics[scale=0.63]{res/Methodology/usecase_diagram}
	\caption{A usecase diagram for Monkey Office}
	\label{fig:usecase-diagram}
\end{figure}

Figure \ref{fig:usecase-diagram} captures system functionalities and requirements represented as \gls{uml} use case diagram.
Table \ref{tbl:actor-description} describes who these type of users are.
From table \ref{tbl-usecase:reject-document} to table \ref{tbl-usecase:attach-conclusive-documents} elaborates each use cases in greater detail.

\begin{table}
	\centering
	\caption{Type of user and description}
	\label{tbl:actor-description}
	\begin{tabular}{|l|L{7.5cm}|}
		\hline
		Actor & Description \\
		\hline
		Dean & A headmaster of \gls{ic}. \\
		Deputy dean & Dean's assistant who could also be a dean's representative. \\
		Assistant of deputy dean & Deputy dean's assistant who deals with documents that is not critical to \gls{ic}. \\
		Staff & \gls{ic}'s employee in academic department (refer to bottom-most hierarchy in figure \ref{ic-org-sturcture}) \\
		User & A regular user who have an account provided by an administrator to access the system. \\
		\hline
	\end{tabular}
\end{table}

\newcommand{\alreadylogin}{Have already logged in to the system}
\newcommand{\allICPersonel}{
	\begin{itemize}
		\item Dean
		\item Deputy dean
		\item Assistant of deputy dean
		\item Staff
	\end{itemize}
}

\begin{table}
	\centering
	\caption{Use case: Reject Document}
	\label{tbl-usecase:reject-document}
	
	\begin{tabular}{|l|L{9cm}|}
		\hline
		\textbf{Use Case Name} & Reject Document \\
		\hline
		
		\textbf{Description} & Decline the received document and halt the process of this document. \\
		\textbf{Primary Actor} & \allICPersonel \\
		\textbf{Pre-condition} & \alreadylogin. \\
		\textbf{Post-condition} & 
		\begin{itemize}
			\item The system notifies user for a rejected document
			\item The rejected document disappears from the list of pending documents
		\end{itemize} \\
		\textbf{Trigger} & Read the document \\
		\textbf{Basic flow} & 
		\begin{enumerate}
			\item Click a reject button on a designated document.
			\item Click \enquote{Yes} when system asks for a confirmation.
		\end{enumerate} \\
		\hline
	\end{tabular}
\end{table}

\begin{table}
	\centering
	\caption{Use case: Approve Document}
	\begin{tabular}{|l|L{9cm}|}
		\hline
		\textbf{Use Case Name} & Approve Document \\
		\hline
		
		\textbf{Description} & Accept the received document and let it continue to the next execution process. \\
		\textbf{Primary Actor} & \allICPersonel \\
		\textbf{Pre-condition} & \alreadylogin. \\
		\textbf{Post-condition} & 
		\begin{itemize}
			\item The system notifies user for a approve document
			\item The approved document disappears from the list of pending documents
		\end{itemize} \\
		\textbf{Trigger} & Read the document \\
		\textbf{Basic flow} & 
		\begin{enumerate}
			\item Click an approve button on a designated document.
			\item Click \enquote{Yes} when system asks for a confirmation.
		\end{enumerate} \\
		\hline
	\end{tabular}
\end{table}

\begin{table}
	\centering
	\caption{Use case: Print Document}
	\begin{tabular}{|l|L{9cm}|}
		\hline
		\textbf{Use Case Name} & Print Document \\
		\hline
		
		\textbf{Description} & Print paper documents from a printer. \\
		\textbf{Primary Actor} & Staff \\
		\textbf{Pre-condition} & \alreadylogin. \\
		\textbf{Post-condition} & Physical document is print out from the printer successfully. \\
		\textbf{Trigger} & Open the document \\
		\textbf{Basic flow} & 
		\begin{enumerate}
			\item Click \enquote{Print Document} on the main menu.
			\item Select attachments to be print (if any).
			\item Select a printer.
			\item Configure paper size and number of copies as needed.
			\item Click \enquote{Print} to print the document.
		\end{enumerate} \\
		\hline
	\end{tabular}
\end{table}

\begin{table}
	\centering
	\caption{Use case: Search Document}
	\begin{tabular}{|l|L{9cm}|}
		\hline
		\textbf{Use Case Name} & Search Document \\
		\hline
		
		\textbf{Description} & Ask the system to find documents. \\
		\textbf{Primary Actor} & User \\
		\textbf{Pre-condition} & \alreadylogin. \\
		\textbf{Post-condition} & The system lists documents according to search query \\
		\textbf{Trigger} & Finding specific documents \\
		\textbf{Basic flow} & 
		\begin{enumerate}
			\item Enter the document's identification code or the name of the form.
			\item Adding specific searching criteria such as date creation or document's status.
			\item Click \enquote{Search}.
		\end{enumerate} \\
		\hline
	\end{tabular}
\end{table}

\begin{table}
	\centering
	\caption{Use case: Attach Other Required Documents}
	\begin{tabular}{|l|L{9cm}|}
		\hline
		\textbf{Use Case Name} & Attach Other Required Documents \\
		\hline
		
		\textbf{Description} & Some forms require including attachment files in order to continue the process. \\
		\textbf{Primary Actor} & User \\
		\textbf{Pre-condition} & \alreadylogin. \\
		\textbf{Post-condition} & The form includes all uploaded documents \\
		\textbf{Trigger} & The form requires attachments \\
		\textbf{Basic flow} & 
		\begin{enumerate}
			\item Browse files on a local machine and select files that the form requires.
			\item Upload them. Wait for them to all finish.
			\item Click \enquote{Next} to continue the execution.
		\end{enumerate} \\
		\hline
	\end{tabular}
\end{table}

\begin{table}
	\centering
	\caption{Use case: Attach Conclusive Documents}
	\label{tbl-usecase:attach-conclusive-documents}
	\begin{tabular}{|l|L{9cm}|}
		\hline
		\textbf{Use Case Name} & Attach Conclusive Documents \\
		\hline
		
		\textbf{Description} & Includes the document that has physical signatures to certify it as a hard evidence. \\
		\textbf{Primary Actor} & Staff \\
		\textbf{Pre-condition} & 
		\begin{itemize}
			\item \alreadylogin
			\item All required signatures present.
		\end{itemize} \\
		
		\textbf{Post-condition} & The form has this document in its attachments \\
		\textbf{Trigger} & The form has completed execution inside its workflow. \\
		\textbf{Basic flow} & 
		\begin{enumerate}
			\item Select a pending form.
			\item Scan the physical document.
			\item Upload the scanned document as an attachment.
		\end{enumerate} \\
		\hline
	\end{tabular}
\end{table}

\section{The Big Picture}


\section{Storing Digital Documents}
Storing documents mean saving documents uploaded by users and create a metadata so that the system can access them.
Sections below are challenges presented about this problem.

\subsection{what metadata should be stored alongside the document}
The purpose of storing metadata is to create an abstract concept of the document.
According to requirements and use cases, the document should contain attachments and other related documents.
The document can be official or unofficial documents.
Both official and unofficial documents are the same except for one characteristic.
Official documents has a customized identification code as mentioned in table \ref{tbl-doc-subtype}.
Unofficial documents has only regular identification number starting at one and increment by one for each document.
Therefore, the parent document have to store a list of attachments.
That is a list referring to itself.
A class diagram in figure \ref{fig:doc-template} shows relationships between each unit.

\begin{figure}
	\caption{Document template creation class diagram}
	\label{fig:doc-template}
	\includegraphics[scale=0.5]{res/software-design/document_templating}
\end{figure}

\textit{Document} defines attributes that represent a document.
Table \ref{uml:document-templating} describes what each attribute is, its data type, and its functionality from figure \ref{fig:doc-template}.
\begin{longtable}{lL{3cm}L{7.5cm}}
	\caption{Description of each attribute in \textit{Document} class}
	\label{uml:document-templating} \\
	\hline
	attribute & Data type & Description \\
	\hline
	\endfirsthead
	
	\hline
	attribute & Data type & Description \\
	\hline
	\endhead
	
	\hline \multicolumn{3}{r}{{Continued on next page}} \\ \hline
	\endfoot
	
	\hline \hline
	\endlastfoot
	
	owner & String & The original user who create the document. \\
	
	dateCreate & Date & A date when this record is created. \\
	
	name & String &
	The filename including extension that is going to appear on the page.
	It is the original filename that user uploaded.
	If it is a by-product from the workflow, the system automatically generates it. \\
	
	is\_auto\_generate & Boolean &
	A boolean indicates whether this document is created by the system. \\
	
	status & String & 
	Indicates the current states document going through the workflow. \\
	
	filepath & String &
	Absolute path to the file stored in a server. \\
	
	relate2Docs & array of \textit{Document} &
	Additional documents that involved with this document.
	If this document refer to other documents, they will be shown here. \\
	
	attachments & array of \textit{Document} &
	Other required documents or other dependent documents. \\
\end{longtable}

A schema has to be defined first in order to record document's metadata to the database.
The schema relies on a \textit{mongoose.Schema} package.
It provides utility for defining fields and data types suitable for saving to the database.
As the requirement states that attachments are also document.
\textit{relate2docs} and \textit{attachments} are fields that store additional documents.
They should have \textit{Document} data type.
Setting them this way provides a benefit that all metadata about attachments and related documents can be queried from the same object.

\subsection{where digital files should reside in the the server}

\section{Retrieving Digital Documents}

\section{Rejecting and Approving Digital Documents}

\section{Mock-up}

\section{Development Tools}